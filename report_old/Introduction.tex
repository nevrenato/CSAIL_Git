\section{Introduction}

This project aims to build a precise model of how the git works. \par
The purpose of this model is to comprehend better git and 
check some properties that the model does
(not) guarantees.\par

\subsection{A brief explanation of git}

Git is just one of many \Gls{version control system} available. 
They are 
distinguished by two main types: Centralized and Distributed.
Git it's in the later. The major difference between git and any
other VCS, is that while others save differences between files, git saves
snapshots. \par
A Git project usually has three main sections: Repository, Index and
Working Directory. Where the later it's just a subset of the filesystem. \par
Git was created in 2005 by the Linux community, 
with the purpose of replacing BitKeeper. Some of it's
goals are \cite{progit}: 

\begin{enumerate}
	\item Speed
	\item Simple design
	\item Strong support for non-linear development
	\item Fully distributed
	\item Able to handle large projects like the Linux Kernel 
	efficently (speed and data size)
\end{enumerate}

\subsection{Why the verification of git is important}

Nowadays, the software projects are getting bigger, and in most
cases it is needed a version control system. Thus, git is being
more and more used. However, can we trust in git to manage projects
of big value ? We don't know the answer, but it's true that we can be
more confident in git if we make some kind of verification on it. Because
there are no known formal/semiformal verifications of git, we try to fill
part of that hole with this project.
