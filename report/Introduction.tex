\section{Introduction}

A \Gls{version control system} is a tool that records changes on your
files over time. The main idea is to keep track of the changes on your
files and to retrieve them later. There are mainly three kinds of 
\Gls{version control system}, local, centralized and distributed. On
a local VCS it is not possible to collaborate with other users and
the structure that keeps track of the files is kept only locally. The
need for collaborations with other users brought the centralized VCS.
On the centralized version control systems exists a server that keeps
track of the changes on the files. Each user pull and checks out files
from that central server. This approach has mainly two problems. The
first is to have a single point of failure. If a server goes down
during a certain time, nobody can check out or pull updates. The
second problem is when you are not connected to the network. You
cannot pull or commit your changes. To solve this problems, the
distributed VCS appear. Here each user keeps a mirror of the
repository. There is not a central server. An user can always commit,
and when connected, can pull/checkout the changes.

This project aims to build a precise model of how the git works. \par
The purpose of this model is to better comprehend git and 
check some properties that the model (not) guarantees. \par
Also, using this model as basis, a manual will be build. This one,
has the advange of being based on a semiformal specification.

\subsection{A brief explanation of git}

Git is just one of many \Gls{version control system} available. 
They are 
distinguished by two main types: Centralized and Distributed.
Git it's in the later. The major difference between git and any
other VCS, is that while others save differences between files, git saves
snapshots. \par
A Git project usually has three main sections: Repository, Index and
Working Directory. 
Where the later it's just a subset of the filesystem. \par
Git was created in 2005 by the Linux community, 
with the purpose of replacing BitKeeper. Some of it's
goals are \cite{progit}: 

\begin{enumerate}
	\item Speed
	\item Simple design
	\item Strong support for non-linear development
	\item Fully distributed
	\item Able to handle large projects like the Linux Kernel 
	efficiently (speed and data size)
\end{enumerate}

\subsection{Why the verification of git is important}

Nowadays, the software projects are getting bigger, and in most
cases it is needed a version control system. Thus, git is being
more and more used. However, can we trust in git to manage projects
of big value ? We can be
more confident in git if we make some kind of verification on it. Because
there are no known formal/semiformal verifications of git, we try to fill
part of that hole with this project.

\subsection{Alloy}
We will use Alloy as the specification language used for the model.
