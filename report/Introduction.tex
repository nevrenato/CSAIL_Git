\section{Introduction}
\subsection{What is a VCS system?}
% VERSION CONTROL SYSTEM
A version control system is a tool that records changes on your
files over time. The main idea is to keep track of the changes on your
files and to retrieve them later. There are mainly three kinds of 
version control system, local, centralized and distributed. On
a local VCS it is not possible to collaborate with other users and
the structure that keeps track of the files is kept only locally. The
need for collaborations with other users brought the centralized VCS.
On the centralized version control systems exists a server that keeps
track of the changes on the files and each user pulls and checks out files
from this server. This approach has mainly two problems. The
first is to have a single point of failure, if a server goes down
during a certain time, nobody can check out or pull updates. The
second problem is when you are not connected to the network, you
cannot pull or commit your changes. To solve this problems, the
distributed VCS appear. Here each user keeps a mirror of the
repository and there is not a central server. An user can always commit,
and when connected, can pull/checkout the changes.\\
\subsection{What is Git?}
%GIT
Git is a distributed VCS. It was created in 2005 by Linus Torvalds,
for the Linux kernel development. The main difference when comparing to
others distributed VCS is that git keeps snapshots of how your system
looks likes on a certain moment in time, instead of keeping track of
changes.\\
\subsection{Our motivation for this manual}
%OUR PROJECT
Nowadays git is having great success, it is efficient, fast and very
easy to use. The problem here is that, most of the users do not dive
into the git internals, they are just happy that it works. The ones
that actually try to go deep into git to see how it works and why git
behaves in a specific way, are fronted with the lack of formal or even
informal specifications.  So, the purpose of this project is to
formally model the git core using a tool called 
Alloy, analyse and model some git
operations with that tool, and then check which properties the model does (not)
guarantee. This manual presents to the reader a informal description of
git internals, the description of some operations and it ends with
some properties that the model guarantee.\par
%ALLOY
Alloy is a declarative specification language, used for describing structures in 
a formal way. After modeling the structures, it is possible to visualize 
instances of them. \par

Concluding, the difference that this manual has from the others, is
that, before building it, we tried to understand and model 
the git concepts from a formal perspective using Alloy. As result of that
we think that our understanding on some key parts of Git, is more precise
and rigorous, than most manuals known. Thus, we try with this manual
to pass the knowledge obtained. \par 

\subsection{How this manual is structured}
%report organization
This reports starts by explaining the organization of git, how git is
structured. Then we cover each component of git. We begin with the
object model, following the index and working directory. For matters
of time and complexity we focus our model on the index and object
model. After the presentation of the structure we give the informal
specification of some operations. We finish the report with the
analysis of some properties.

% can we trust in git to manage projects of big value ? 

