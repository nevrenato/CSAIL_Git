\section{Introduction}
\subsection{Git}
% VERSION CONTROL SYSTEM
A version control system is a tool that records changes on your
files over time. The main idea is to keep track of the changes on your
files and to retrieve them later. There are mainly three kinds of 
version control system, local, centralized and distributed. On
a local VCS it is not possible to collaborate with other users and
the structure that keeps track of the files is kept only locally. The
need for collaborations with other users brought the centralized VCS.
On the centralized version control systems there exists a server that keeps
track of the changes on the files and each user pulls and checks out files
from this server. This approach has mainly two problems. The
first is to have a single point of failure, if a server goes down
during a certain time, nobody can check out or pull updates. The
second problem is when you are not connected to the network, you
cannot pull or commit your changes. To solve this problems, the
distributed VCS appear. Here each user keeps a mirror of the
repository and there is not a central server. An user can always commit
and when connected can pull/checkout the changes.\\

%GIT
\emph{Git} is a distributed VCS. It was created in 2005 by Linus Torvalds,
for the Linux kernel development. The main difference when comparing to
others distributed VCS is that \emph{git} keeps snapshots of how your system
looks likes on a certain moment in time instead of keeping track of
changes.\par
\subsection{Motivation}
%OUR PROJECT
Nowadays \emph{git} is having great success, it is efficient, fast and
in the begin it looks easy to understand and to use. When
the users start diving into git, they are faced with a behavior that
nobody can explain very well. There are not any formal or even informal
specification of the operations, like what are the pre-requisites and
what is the result of performing an operation. So normally the users
when fronted with something they do not understand, they avoid it next
time.\par
The purpose of this project is to
formally model the \emph{git} core using a tool called 
Alloy, analyse and model some \emph{git}
operations and then check which properties the model does (not)
guarantee. This manual presents to the reader semi-formal description of
\emph{git} internals, the description of some operations and it ends with
some properties that the model guarantee.\par
%ALLOY
Alloy is a declarative specification language, used for describing structures in 
a formal way. After modeling the structures, it is possible to visualize 
instances of them. \\

So, the difference that this manual has from the others, is
that, before writing it, we tried to understand and model 
the \emph{git} concepts from a formal perspective using Alloy. As result 
we think that our understanding on some key parts of \emph{git}, is more precise
and rigorous, than most known manuals. Thus, we try with this manual
to pass the knowledge obtained. \par 

\subsection{Manual Structure}
%report organization
This reports starts by explaining the organization of \emph{git}, how
\emph{git} is
structured and then we cover each component of \emph{git}, working directory,
index and repository. For matters
of time and complexity we focus our model on the index and object
model. After the presentation of the structure we give the semi-formal
specification of some operations. We finish the report with the
analysis of some properties.

% can we trust in git to manage projects of big value ? 

