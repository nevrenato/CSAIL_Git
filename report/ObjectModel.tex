\section{Modelling Git Object Model}

For each part of a textual 
specification we will associate an Alloy specification.
The specification comes from \cite{gitComm}. \par

%%%
"All the information needed to represent the history
of a project is stored in files referenced by a 
40-digit "object name"...."


\begin{lstlisting}
sig Sha{}
\end{lstlisting}
%%%%

"..and there are four different types of objects: "blob",
"tree", "commit", and "tag"."

\begin{lstlisting}
abstract sig Object {
	namedBy : one Sha
}

sig Blob extends Object{}

sig Tree extends Object {}

sig Commit extends Object{}

sig Tag extends Object{}
\end{lstlisting}

%%%%
"A "blob" is used to store file data - it is generally
a file."

\begin{lstlisting}
sig Blob extends Object{}
\end{lstlisting}

%%%%
"A "tree" is basically like a directory - it references a bunch
of other trees and/or blobs..."

\begin{lstlisting}
sig Tree extends Object {
	references : set (Tree+Blob)
}
\end{lstlisting}

%%%%

"A "commit" points to a single tree...."

\begin{lstlisting}
sig Commit extends Object{
	points : one Tree
}

\end{lstlisting}
%%%%

"A "tag" is a way to mark a specific commit..."

\begin{lstlisting}
sig Tag extends Object{
	marks : one Commit
}
\end{lstlisting}

%%%%

Next, as the book \cite{gitComm} says, a "tree" acts
like a directory, so it or it's descendents cannot point
to itself.

\begin{lstlisting}
no ^references & iden 
\end{lstlisting}

"...two "trees" have the same SHA1 name if and only if their
contents (including, recursively, the contents of all subdirectories)
are identical."

\begin{lstlisting}
all t,t' : Tree | t.namedBy = t'.namedBy <=> t.references = t'.references
\end{lstlisting}

%%%%
"What that means to us is that is virtually impossible to find to 
different objects with the same name"

\begin{lstlisting}
namedBy.~namedBy - (Tree->Tree) in iden
\end{lstlisting}


