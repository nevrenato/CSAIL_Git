\section{Git Object Model}

In this section, we will specify the git object model, using as
basis two manuals, \cite{gitComm} and \cite{progit}. \par
After the specification of the git object model we can start
to specify it's operations. \par

\subsection{Identification of Objects}
All objects are identified by a sha defined by their contents. \par
"All the information needed to represent the history
of a project is stored in files referenced by a 
40-digit {\bf "object name"}..." (page 7) \par

However Alloy has a mechanism to uniquely identify atoms, so 
we can use it instead of the sha. \par

\subsection{The four types of Objects}
The objects are defined as in \cite{gitComm} (page 7) wich says: 
"...and there are four different types of objects: blob,
tree, commit, and tag."
Also: "A blob is used to store file data - it is generally a file" 
\cite{gitComm} (page 8). \par
For trees: "A tree is basically like a directory - {\bf it references a bunch
of other trees and/or blobs}..." (page 8) \par 

Now for the commit:  
"As you can see,a commit is defined by : 
parent(s) : {\bf The SHA1 name of some number of commits which
represent the immediately previous step(s) in the 
history of the project}..."
"...merge commits may have more than one. A commit with no 
parents is called a root commit, and represents the 
initial revision of a project. {\bf Each project must have at
least one root. A project can also have multiple roots,
though that isn't common (or necessarily a good idea)}". \cite{gitComm} (page 12)
A commit also points to a certain tree that represents the state of the repository
at a given state. \par

\section{Branches in Git}
One of the advantages of git compared to others VCS is the operation
of creating a new branch. While in others VCS, a copy of the hole project is
done each time we create new branch, in git, only a new pointer is created.

"A branch in Git is {\bf simply a 
lightweight movable pointer to one of these commits}." \cite{progit} (pag 39) \par

Also:"The special pointer called {\bf Header 
points to the branch we are working on}". \cite{progit} (pag 40) \par


\section{Index in Git}
The definition of Index:
"...staging area between your working directory and your
repository. You can use the index to {\bf build up a set of 
changes that you want to commit together}. When you create
a commit, {\bf what is committed is what is currently in the
index, not what is in your working directory.}"
\cite{gitComm} (page 17). \par

Also : ``The index {\bf contains all the information necessary to generate a single
(uniquely determined) tree object}'' \cite{gitComm} (pag 121). \par

\section{Specification of files}

The alloy specification
