\section{Modelling Git Static Object Model}

In this section we will show the specification in Alloy of the
git object model. This will be mostly based on the textual
specification present in \cite{gitComm}. \par
Thus, we will divide this section in conditions that are specified 
in \cite{gitComm} and conditions that we didn't found explicity
in the book but that are needed for the proper functioning of
git.

\subsection{Explicit specifications}

"All the information needed to represent the history
of a project is stored in files referenced by a 
40-digit {\bf "object name"}..." (page 7)

\begin{lstlisting}
sig Sha{}
\end{lstlisting}

"...and there are four different types of objects: "blob",
"tree", "commit", and "tag". " (page 7)

\begin{lstlisting}
abstract sig Object {
	namedBy : Sha
}
sig Blob extends Object{}
sig Tree extends Object {}
sig Commit extends Object{}
sig Tag extends Object{}
\end{lstlisting}

"A "tree" is basically like a directory - {\bf it references a bunch
of other trees and/or blobs}..." (page 8)

\begin{lstlisting}
sig Tree extends Object {
	references : set (Tree+Blob)
}
\end{lstlisting}

"A "commit" {\bf points to a single tree}...." (page 8)

\begin{lstlisting}
sig Commit extends Object{
	points : Tree
}
\end{lstlisting}

"A "tag" is a way to { \bf mark a specific commit}..." (page 8)

\begin{lstlisting}
sig Tag extends Object{
	marks : Commit
}
\end{lstlisting}

As the book \cite{gitComm} says, a "tree" acts
like a directory, so we know that it or it's 
descendants cannot point to themselves.
\begin{lstlisting}
no ^references & iden 
\end{lstlisting}

"...two "trees" have the { \bf same SHA1 name if and only if their
contents (including, recursively, the contents of all subdirectories)
are identical.} " (page 11)

\begin{lstlisting}
all t,t' : Tree | 
   t.namedBy = t'.namedBy <=> t.references = t'.references
\end{lstlisting}

"What that means to us is that is virtually {\bf impossible to find to 
different objects with the same name}". (page 7) \par
However we know that there is an exception. 
There are cases (defined above)
that allow different trees to have same name. 

\begin{lstlisting}
namedBy.~namedBy - (Tree->Tree) in iden
\end{lstlisting}


\subsubsection{Commit structure}

Again from \cite{gitComm}. \par 
"As you can see,a commit is defined by :
... \par
parent(s) : {\bf The SHA1 name of some number of commits which
represent the immediately previous step(s) in the 
history of the project}. The example above has one parent;
merge commits may have more than one. A commit with no 
parents is called a "root" commit, and represents the 
initial revision of a project. {\bf Each project must have at
least one root. A project can also have multiple roots,
though that isn't common (or necessarily a good idea)}". (page 12)

\begin{lstlisting}

sig Commit extends Object{
  points : Tree,
  parent : set Commit
}
some sig RootCommit extends Commit{}
fact {
  no ^parent & iden
  no RootCommit.parent
  Commit in *parent.RootCommit
}

\end{lstlisting}

\subsection{What we assume for the proper functioning of git}

A commit must point only root trees, unless we want
losses of information.
\begin{lstlisting}
no Commit.points & Tree.references    
\end{lstlisting}

In the object model we don't have orphan Blobs. 
\begin{lstlisting}
Blob in Tree.references
\end{lstlisting}

The same for Trees.
\begin{lstlisting}
Tree in Tree.references + Commit.points
\end{lstlisting}

Sha's that don't point to anywhere are useless.
\begin{lstlisting}
Sha in Object.namedBy
\end{lstlisting}

We can't have the same tree shared by commits. Git
doesn't work that way ("it stores a snapshot of what all 
the files in your project looks like..." \cite{gitComm} (page 7)).
\begin{lstlisting}
points.~points in iden 
\end{lstlisting}

A tree has at most one parent. If this is not true, than there 
is inconsistency in the git "filesystem".
\begin{lstlisting}
all t:Tree | lone references.t
\end{lstlisting}

