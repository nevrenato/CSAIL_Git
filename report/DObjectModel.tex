\section {Modelling Git Dynamic Object Model}
The static object model, only by itself, 
is not very usefull. And so
the model would be much more interesting if we specify 
the most important git operations. That said in this
section we will focus on modelling the most important
and critical of them. \par

In order to understand them, the definition
of the git Working Directory and git Index is in debt.

\subsection{Introducing branches}
Definition of branch: "A branch in Git is {\bf simply a lightweight movable pointer to one of these commits}." \cite{progit} (pag 39)\\
A branch in git, is a pointer for a commit. We are always working on a certain branch, and all the commits made on that branch are independent from the others branch until we make a merge. A new branch can be created using \emph{git branch branch\_name}. The special pointer called Header points to the branch we are working on. We can go to another branch using \emph{git checkout branch\_name}.\\



Trying to understand branches using git:
\begin{itemize}
   \item if we modify a tracked file, we cannot change branch if we do not add that file to the index;
   \item if we add a tracked and modified file to the index and we change branch, that file will be presented as it was on the other branch;
   \item if we create a new file on a branch and then we change branch (without adding the file to the index) that file will be present;
   \item if we create a new file, add it to the index, change branch, commit, go back to the other branch, that file will not be visible;
   \item master branch can be deleted;
\end{itemize}

We can conclude that the index is not connected to a branch. It is something is common to all branches.
\subsection{Working Directory and Index}

The definition of Working Directory comes as follows:
"... simply a {\bf temporary checkout place} where you can 
modify the files {\bf until your next commit}."
\cite{gitComm} (pag 17). \par

To complement the definition above, git will only
track files that existed in the last commit, thus
new ones will not appear in this directory. \par 
So,
the Working Directory, does nothing more than telling
us wich files have been modified since the last commit.

\begin{lstlisting}
sig File{}
sig WorkingD {
	contents : set File
}
\end{lstlisting}

Now for the definition of Index:
"...staging area between your working directory and your
repository. You can use the index to {\bf build up a set of 
changes that you want to commit together}. When you create
a commit, {\bf what is committed is what is currently in the
index, not what is in your working directory.}"
\cite{gitComm} (page 17). So, the Index, in a certain
manner, points to a subset
of files that are present in the Working Directory, and
is this subset of files that wil be commited. 

\begin{lstlisting}
sig Index {
	toCommit : set File
}
\end{lstlisting}

However
there will be a few exceptions!(present in the last parapgrah of the sections
git add command and git rm command)\par

We can see  
that a commit depends on the contents of git Index, as the
last depends on the Working Directory. So it will be important
to add those definitions to our model when defining the commit operation
.But, a question arises.
What is the pratical difference between the Index and the Working
Directory? Shouldn't all that we modify in the repository be 
commited? We will see that next. 

\subsubsection{The git add command}

This command is one of the most used in git. \cite{gitComm} (page 26)
tells us that
"git add is {\bf used both for new and newly modified files},
and in both cases it takes a snapshot of the given files
and {\bf stages that content in the index, ready for inclusion
in the next commit}." \par 
When doing a git add of a newly
added file, that
addition will not appear in the Working Directory, only in the
Index. 
\subsubsection{The git rm command}

This one is generally less used. It's usefullness can be
seen in the following : \cite{Chacon:2009:PG:1618548}
(page 21) says "To remove a file from Git, you have to remove it
from your tracked files (more accurately, {\bf remove it from your
staging area}) and them commit. The {\bf git rm command does that
and also removes the file from your working directory}...". \par
Note that the working directory refered in the citation is not
the git Working Directory, it's the directory of the filesystem itself.
In other words a 'rm <file>' is present inside the 'git rm <file>'. \par
When using git rm
that modification will only appear in the Index. Unless, before
the git rm we delete the file from the filesystem (e.g. rm ). \par

