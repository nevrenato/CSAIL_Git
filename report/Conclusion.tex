\chapter{Conclusion}
In this project we started by modeling the git internals using Alloy
\cite{Jackson:2006:SAL:1146359}. When we reached
a stable model we started modeling some operations, as for example, git
add, git rm, git commit, etc. Such model and such
operations were modeled based on \cite{gitComm}, \cite{progit} git man
pages and using git to perform tests. After some operations were
modeled, we checked for some properties, as for example, invariant
preservation, idempotence, etc.\\

When the operations were being modeled many
difficulties were faced due to the lack of precise descriptions. 
Many tests had to be performed to find a general behavior for each
operation. During those tests a bug in git was found. In
\ref{bugappend}, we describe a
sequence of operations to reach a state were information is lost. This
bug was reported in the \href{mailto:git@vger.kernel.org}{Git Mailing
List} and it was proof that a lot of people are confused with the
operations' behavior. In the begin we received some answers saying it
was the normal behavior and then the \emph{git} maintainer assumed it
looked like a bug.\\

This report as well as
\href{http://nevrenato.github.com/CSAIL_Git/}{http://nevrenato.github.com/CSAIL\_Git/},
show a detailed description from the work we have done analysing the
model we built. We tried to write it without using Alloy, so
everybody can read it, even if, they have never heard about Alloy.\\

The time we had to work on this project was not enough (not even close) to
go through the whole \emph{git}. Some choices were made trying to
simplify the model and choosing the most important operations. The
model is stable, so as future work, we suggest to model some more
operations.
