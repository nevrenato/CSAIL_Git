\chapter{Conclusion}
In this project we started by modeling the git internals using Alloy
\cite{Jackson:2006:SAL:1146359}. When we reached
a stable model, we started modeling the most used operations:
add, rm, commit, and checkot. Such model was defined based on the
documentation in some popular git manuals \cite{gitComm,progit}, the git man pages,
and by performing additional tests using git. After some operations were
modeled, we checked for some properties, like invariant
preservation or idempotence.\\

We had many difficulties when modeling the operations due to the lack of precise descriptions. 
Many tests had to be performed to find a general behavior for each
operation. During those tests a bug in git was found. In appendix
\ref{bugappend} and \ref{bugappend2}, we describe a
sequence of operations to reach a state were information is lost in the
checkout operation. This
bug was reported in the \href{mailto:git@vger.kernel.org}{Git Mailing
List} and by the replies we received many people seemed confused about
the intended behavior of this operation. First, we received some answers saying it
was the normal behavior and then the \emph{git} maintainer concluded that
it was indeed a bug.\\

This report, as well as
\href{http://nevrenato.github.com/CSAIL_Git/}{http://nevrenato.github.com/CSAIL\_Git/},
show a detailed description from the work we have done analysing the
model we built. We wrote the report without presenting the Alloy
formalization, so that readers not familiar with Alloy can also
benefit from it. The full Alloy model can be found in the above URL.\\

The time we had to work on this project was not enough (not even close) to
go through the whole \emph{git}. Some choices were made trying to
simplify the model and choosing the most important operations. The
model is considered by us, to be stable and ready to be extended with
more operations.\\
