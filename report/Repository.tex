\section{The Repository}
\subsection{Git Object Model}

As we have presented before git keeps snapshoots of how your system
looked like on a certain moment in time. This moment in time is
represented in git by a commit. A commit points to a tree, that
represents the structure of your system on that moment. So a tree
contains others trees and/or blobs. In git the files are not kept. 
What git keeps is a blob that represents the content of files. The
relation between the path of a file and the content of that path 
is kept on the trees corresponding to that path.

We will show the 
specification of the git object model, using as
basis two manuals, \cite{gitComm} and \cite{progit}. \par
Only the key parts of the model will be presented. \par

\subsection{Identification of Objects}

All objects are identified by a sha defined by their contents. \par
"All the information needed to represent the history
of a project is stored in files referenced by a 
40-digit {\bf "object name"}..." (page 7) \par
We don't need to specify this, as Alloy already can uniquely
identify atoms, so we can reutilize that functionality. \par


\subsubsection{The four types of Objects}

The objects are defined as in \cite{gitComm} (page 7) wich says: 
"...and there are four different types of objects: blob,
tree, commit, and tag."
Also: "A blob is used to store file data - it is generally a file" 
\cite{gitComm} (page 8). \par
For trees: "A tree is basically like a directory 
- {\bf it references a bunch
of other trees and/or blobs}..." (page 8) \par 

\begin{lstlisting}
abstract sig Object {
	objects : set State
}

sig Blob extends Object {}

sig Tree extends Object {
	contains : Name -> lone (Tree+Blob)
}
\end{lstlisting}

The tag will be discarded, as it doesn't affect the main operations
of git. \par 

Now for the commit:  
"As you can see,a commit is defined by : 
parent(s):{\bf The SHA1 name of some number of commits which
represent the immediately previous step(s) in the 
history of the project}..."
"...merge commits may have more than one. A commit with no 
parents is called a root commit, and represents the 
initial revision of a project. {\bf Each project must have at
least one root. A project can also have multiple roots,
though that isn't common (or necessarily a good idea)}". 
\cite{gitComm} (page12) 
A commit also points to a certain tree that represents the state of the
repository at a given state. \par

\begin{lstlisting}
sig Commit extends Object {
	points : Tree,
	parent : set Commit,
}
\end{lstlisting}

\section{Branches in Git}
One of the advantages of git compared to others VCS is the operation
of creating a new branch. While in others VCS, a copy of the hole project is
done each time we create new branch, in git, only a new pointer is created, as
said in the next quotations:
"A branch in Git is {\bf simply a 
lightweight movable pointer to one of these commits}." \cite{progit} (pag 39) \par
"The special pointer called {\bf Header 
points to the branch we are working on}". \cite{progit} (pag 40) \par

\begin{lstlisting}
sig Branch {
	marks : Commit one -> State
	branches : set State,
	head : set State
}
\end{lstlisting}

\section{Specification of files}

In order to model Git, we need the notion of a file. In our view a file should have
a path associated and a content. The path (along with it's parents)
will uniquely determine the full name of a file
(e.g. /x/y/example.txt), the content will be just a Blob.

\begin{lstlisting}
	sig File {
		path : Path,
		blob : Blob
	}

	sig Path {
		pathparent : lone Path,
		name : Name
	}
\end{lstlisting}

\begin{figure}[h!] 
	\caption{A typical example of a file, where is name is /Name1/Name0 }
	\centering
	\includegraphics[scale=0.65]{images/image1.png}
\end{figure}
\pagebreak


\section{Index in Git}

The definition of Index:
"...staging area between your working directory and your
repository. You can use the index to {\bf build up a set of 
changes that you want to commit together}. When you create
a commit, {\bf what is committed is what is currently in the
index, not what is in your working directory.}"
\cite{gitComm} (page 17). \par

Also : ``The index {\bf contains all the information necessary to generate a single
(uniquely determined) tree object}'' \cite{gitComm} (pag 121). \par

Thus, what is in the index for a given state, is just a set of files.

\begin{lstlisting}
	sig File {
		path : Path,
		blob : Blob,
		index : set State
	}

\end{lstlisting}

\begin{figure}[h!] 
	\caption{A typical example of a file (file1) that is in the index}
	\centering
	\includegraphics[scale=0.65]{images/image2.png}
\end{figure}
\pagebreak

