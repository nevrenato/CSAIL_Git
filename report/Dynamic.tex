\section {Modelling Git Dynamic Object Model}

The static object model, only by itself, 
is not very useful. And so
the model would be much more interesting if we specify 
the some git operations. In order to do that, we need to 
specify on the model, what can change over time.

\subsection{The notion of state}

From now on, all relations with potential to change must
have a state associated. So, a relation
that has some information before an operation, can have
other contents after it. \par
For example :

\begin{lstlisting}
sig Commit extends Object {
	...
	parent : Commit set -> State
}
\end{lstlisting}

We can see that a commit can have a determined set of parents, but, 
after some operation it can have another set of parents. Thus, the state
will define the relation contents in a certain "instant".
\par
The model, now, needs some adaptations to support the notion of state. \par


\subsection{Git operations}

\subsubsection{add}

This command is one of the most used in git. \cite{gitComm} (page 26)
tells us that
"git add is {\bf used both for new and newly modified files},
and in both cases it takes a snapshot of the given files
and {\bf stages that content in the index, ready for inclusion
in the next commit}." \par 
Thus, we can see the add operation, only as the adition of a file 
to the index, and nothing else can change between the two states.
\par

\begin{lstlisting}

	//path needs to be a file
	no pathparent.p

	// a path is added to the index
	objects.s' = objects.s + p.blob
	index.s' = index.s + p

	//all other things are kept
	parent.s' = parent.s
	marks.s' = marks.s
	branches.s' = branches.s
	head.s' = head.s 

\end{lstlisting}

\subsubsection{rm}

The usefullness of rm can be
seen in the following \cite{progit}
(page 21) "To remove a file from Git, you have to remove it
from your tracked files (more accurately, {\bf remove it from your
staging area with git rm}) and then commit.
An important restriction that git imposes when using git rm is used,
it that the
content of the file to remove must be equal to the last commit. \par
So, we can see the rm operation as a removal operation of a file
that is in the index, and it cannot have changed since the last
commit. \par


\subsubsection{commit}

The commit operation is simply explained. 
"A commit is usually created by git commit, which creates a {\bf commit whose parent is 
normally the current HEAD}, and whose {\bf tree is taken from the content currently stored
in the index}" \cite{gitComm} (page 13).
But hard to model. The culprit is the transformation of the index into a tree that is pointed
by the new commit.

\subsubsection{push}
The push operation saves the local changes to a public repository.  

\subsubsection{pull}
The pull operation fetches the repository information, and merges the update with the local
changes.
