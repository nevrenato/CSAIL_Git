
\section{Git operations}
In this section we explore some of the git operations. We try to do
so, in a different way, if comparing with other manuals. Normally the
manuals we see around just explain how to perform an operation. We
try to go further, so in this manual, we have divided our operations in
three parts. In the first part, we give a description about the
operation. In the second part, we give the pre-conditions, or in other
words, we explain in which conditions the operations can be performed.
Finally, in the last part of each operation, we show what is the
result of perform such operation.

\subsection{Add and Remove}
The operations add and remove are used to add and remove content from
index. As we have said before the index contains all the files and
contents to be added on the next commit.\\

\emph{Git} add does not refer directly to the file. Instead it is like
a map from file to content. If we have a file on the index and we
modify it on the working directory, for this modification to be visible
on the next commit, the file has to be added again to the index.\\

The remove operation removes a file from index, so it will not be
present if we perform a commit exactly after the remove. The removed
file besides of being removed from index, it will also be removed from the 
working directory (if it is still exists there).

\subsubsection{Pre requisites}
The add operation has only one pre-condition. The file has to be in
the working directory. It means that a file that is
on the working directory, can be always added to index.

Relatively to the remove operation there are a few conditions that
have to be satisfied for the remove operation be performed, which are: 
\begin{itemize}
\item The file that is being removed is currently in index;
\item The file that is being removed is in the previous commit, exactly with the same
content.
\end{itemize}

The first restriction is quite obvious. It is not possible to remove
a file if it does not exists. The last restriction exists, to avoid
the accidental deletion of file's contents. It is possible to have a
file in index with a certain content that does not exists in the
repository and since the remove operation removes the file from the index
and from the working directory, the content would be lost.  

\subsubsection{Result}
After performing the add operation there is something that will be
always observed. The file will be in index. Besides, if the file was
marked as not merged then the mark goes away. More details about
merged and unmerged files will be given later.\\

The result from performing the remove operation is that the file is
not anymore in the index. Also here the file will not be marked as
unmerged anymore.

\subsubsection{Examples}

The figure \ref{fig:add1}, shows a simple case of adding a file to the 
index. It can be seen, that no changes are made in the repository nor in the working
directory. The only effect is that the index will contain the new
file.\\

\begin{figure}[tp]
   \begin{minipage}[b]{0.5\linewidth}
      \centering
      \includegraphics[width=0.8\textwidth]{images/add1.png}
      \caption{Adding a file}\label{fig:add1}
   \end{minipage}
   \begin{minipage}[b]{0.5\linewidth}
      \centering 
      \includegraphics[width=0.9\textwidth]{images/update1.png}
      \caption{Updating a file}\label{fig:update1}
   \end{minipage}
\end{figure}

Figure \ref{fig:update1}, contains the case of updating the content of a file.
Assuming that the file is already in the index, changing its content in the 
working directory, will not change the content in the index. We need to 
explicitly add the file with the new content in the index. \\ 

Now lets loot at the remove operation. Figures \ref{fig:remove1},\ref{fig:remove2} 
show a simple case of
removing a file from the index. As the reader can see the file will be
removed from the index, but also from the working directory. 
There are no changes on the repository.


\begin{figure}[tp]
   \centering
   \includegraphics[width=0.6\textwidth]{images/remove1.png}
   \caption{Removing a file}\label{fig:remove1}
\end{figure}

\begin{figure}[tp]
   \centering
   \includegraphics[width=0.6\textwidth]{images/remove2.png}
   \caption{Removing a file (alternative)}\label{fig:remove2}
\end{figure}



\subsection{Commit}
The commit operation takes the current index and builds a commit
object. Basically it takes the files from the index, builds a
structure using trees and blobs objects and creates a new commit
object pointing to the tree that corresponds to the project's root
directory. This commit object will have as parent the current commit,
the commit pointed by the branch indicated in the HEAD.
When the first commit is created, a branch called "master" will be created
and it will be marked as the current branch, or in other words, it
will be indicated by HEAD. Also, the first commit of the repository
is called a Root Commit, because it has no parents.

\subsubsection{Pre requisites}
There are three pre-conditions that are required to perform this
operation:
\begin{itemize}
   \item The index is not empty;
   \item The commit object that will be created is different from the
   current commit;
   \item There are not unmerged files.
\end{itemize}
The first pre-condition is obvious, because if we do not have files in
the index, there is anything to commit.\\

The second condition says basically
that we cannot have two commits objects pointing to the same
tree object in a row, or in other words, if we have the same files and
the same content in the index and in the current commit, a commit
operation cannot be performed. \\

The last condition just says that it is not possible to perform a
commit, if there are unmerged files in index. As we have said before,
we will speak about this later.

\subsubsection{Result}
The result of performing a commit can be different if it is the first
commit or if there are already some commit in th repository. 
We start by exposing the observed result when the first commit is
being performed and then we show the case when there are already some
commit in the repository. In the end we show some properties that are
observed in both cases.\\

The first commit object that is being created is a commit with no
parent. It means that it does not have a parent. There cannot be branches
at this moment, because they cannot point to any commit, as there are none to be
pointed, so a new branch called "master" is created. This branch is going to be pointing to 
the commit that has been created. Also the HEAD, that previously was 
not defined will identify this "master" branch.\\

When there are already some commit in the repository, then it is
guaranteed that there are at least one branch and the HEAD is
identifying some existing branch. The branch that is identified by the
HEAD, is pointing now to the commit created. This commit have now as
parent the previous commit (the commit pointed by the branch
identified by the HEAD), unless it is a result from a merge. In this
case the commit must have two parents. The previous commit and the
branch that is being merged.\\

The common properties that are observed are that all the files and
content that were in the index, are now in the current commit with the
structure from the file system. For this new blobs and trees were
probably created. The index keeps exactly the same content.

\subsubsection{Examples}

The figure \ref{fig:commit1} shows a typical process where two files
are added to the index and then a commit is performed. It can be seen
that the new commit reflects the working directory's structure
with these two files. The interesting point is that, because these two files
have the same content, they will share the same
Blob. This is a case where \emph{git} shows its efficiency. \\

The previous commit shows the case, in which the first commit is being
performed. In Figure \ref{fig:commit2} it is possible to see an example 
of a commit operation, in which the only difference from the previous 
commit is a removed file. This figure is 
interesting because it shows which changes are made in the repository when a
new (not RootCommit) is created. The new commit points to the previous commit and
the branch and HEAD identifies to the created commit. As expected the new commit reflects
the working directory's structure for the existing files.\\

Figure \ref{fig:commit_pre} is just a concrete example of the second pre requisite
referred above. In this case the commit cannot be performed.\\

\begin{figure}[hp]
   \centering
   \includegraphics[width=0.8\textwidth]{images/commit1.png}
   \caption{Doing a commit}\label{fig:commit1}
\end{figure}
\begin{figure}[hp]
   \centering
   \includegraphics[width=0.8\textwidth]{images/commit2.png}
   \caption{Doing a second commit}\label{fig:commit2}
\end{figure}

\begin{figure}[tp]
   \centering
   \includegraphics[width=0.6\textwidth]{images/commit_pre.png}
   \caption{An index which does not satisfies the commit
   pre-conditions}\label{fig:commit_pre}
\end{figure}

\subsection{Branch Create and Branch Remove}
Branches in \emph{git} are very efficient. This happens because a
branch in \emph{git} is just a pointer to a commit. So, when a branch
is created or removed there is only a new pointer created or a
pointer that is removed. As we will see later, there are some
restrictions when removing branches. When we are referring to
creating or removing a branch, we mean the operations
\emph{git branch} and \emph{git branch -d}.

\subsubsection{Pre requisites}
When creating a branch there are only two restrictions:
\begin{itemize}
   \item Some commit must exist, and consequently the HEAD reference must point
   to some commit;
   \item There is not any branch with the same name.
\end{itemize}
The second pre-condition is kind of obvious. We cannot create a branch
with the same same name as one already in the repository, otherwise we
would have conflicts. The pre-conditions that could bring some doubts
is the first one. \emph{Git} does not allow the creation of branches
before the first commit is done. When the first commit is done, a
branch called "master" is created and after that it is possible to
create as many branches as the user wishes.\\

When removing a branch there are also some restrictions:
\begin{itemize}
   \item The branch exists;
   \item The HEAD is not identifying the branch;
   \item The commit pointed by the branch is accessible from the
   current commit (the commit pointed by the branch identified by the
   HEAD).
\end{itemize}

The first pre-conditions says that it is only possible to remove a
branch if it exists. The second says, that the branch being removed is
not the one identified by HEAD, otherwise the HEAD would be point from
now on to some branch that does not exists anymore. Thus, there would
not exist a current commit. The last pre-conditions exists to check
the user does not deletes a branch that is not yet merged with the
current branch, or in other words, the user does not delete a branch
that is not accessible from the current commit, through the parent relation. 
If this pre-condition did not exist it would be possible to loose the
commit pointed by the branch, as well as, some commits from the parent
relation.


\subsubsection{Result}
The result of performing this operations are expected. When creating a
branch, the result is a new branching pointing to the same commit as
the branch identified by the HEAD. When removing a branch, the result
is that the branch removed does not exists anymore.

\subsubsection{Examples}
As we wrote before the changes of creating a branch are minimal. The only thing
changing in the repository is the new branch pointing to the current commit, as
it can be seen in figure \ref{fig:branch}. \\

When removing a branch the changes are also minimal. The only thing
changing in the repository is that the 
branch referred must be deleted. If instead of looking at Figure
\ref{fig:branch} from top to bottom, we look at it from bottom to top,
it shows the case when a branch is removed. \\ 

\begin{figure}[!t]
   \centering
   \includegraphics[width=0.6\textwidth]{images/branch.png}
   \caption{Creating a branch}\label{fig:branch}
\end{figure}

\subsection{Checkout}
The checkout operations is used to update the working directory and
index with a set of files from a commit. In this project, we have
explored only the case the checkout operation is used with a branch.
In this case, all the files from the commit pointed by the branch are
exported to the working directory and index. This was the most
difficult operation to explore, because it behaves in a non intuitive
way. We will try to explain it and in the examples we show you a
sequence of operations that causes the \emph{git} to loose
information. This is a proof of the complexity of this operation.

\subsubsection{Pre requisites}
As we have said before, this operation is not intuitive. In our
opinion this operation should have as pre-condition something like
"Everything that is in index should be committed". Instead \emph{git}
tries to relax this pre-condition and we have the following:
\begin{itemize}
   \item The branch we are checking out, exists;
   \item Everything that is in the index has to be in the current commit with the
      same content, except if:
      \begin{enumerate}
         \item The content of a file is the same in the
         current and destination commit (warning is thrown)
         \item Exists a file in the index, and that file does not exists
         neither in the current nor in the destination commit (warning is thrown)
         \item Content of the file in the index is the same as in the
         destination commit (no warning is thrown)
      \end{enumerate}
\end{itemize}

About the pre-condition there are not much to say. If we want to
checkout a branch, the branch should exist.\\

The problem here is the second pre-condition. There are a lot of
exceptions when a file is not in the current commit with the same
content. To simplify the exposition of this pre-condition, lets assume
we have a file f, which is not in current commit, or at least it is not
with the same content. All other files from index are in the current commit, with
the same content. Relatively to f, if at least one of the exceptions enumerated above
is satisfied, then the checkout operation will proceed. Now lets
analyse each one of the exceptions. The first says that, f can 
be in the commit we are checking out with the same content. So, even
if f is not commit, but it is in the commit we are checking out, git
will proceed with the operation. The exception marked as 2, says that
f does not exists neither in the current commit nor in the commit we
are checking out. It means that if f has just been created and a
checkout operation is performed then if the file does not exists in
the commit being checked out, the operation will proceed. The last
exception says that if there exists a file with the same path and the
same content in the commit being checked out, then the operation will
proceed. In this last case does not warns the user.

\subsubsection{Result}
As we have said before, in this manual we have
concentrated our efforts in the index and repository, simplifying as
most as possible the working directory. So, we will just explain what
happens with the index after a checkout operation is performed. We do
not make any description about the files that are in the working directory
but not in index.\\

The first thing the reader should know is that the
HEAD, after a checkout is performed, identifies the branch which was
checked out.\\

There is one more alteration. The index when a checkout is performed
has to be updated to reflect the operation. The result of it is not intuitive. To
understand it lets assume that the checkout operation has not yet been 
performed. Consider three relations from file to content. 
\begin{itemize}
   \item $CA$: Files from the current commit and their
   respective content;
   \item $IA$: Files from index and respective content;
   \item $CB$: Files from the commit we are trying to checkout
   and their respective content.
\end{itemize}
After understanding this three relations it is simpler to understand
what happens with the index. Lets also assume that $(IA-CA)$ is a relation that
contains the files from $IA$ which are not in $CA$ with the same content.
It mean that $(IA-CA)$ will contain the new files and the modified files relatively to
the last commit. Now, consider $CB ++ (IA-CA)$ as being all the files
from $CB$, but if the files are in $(IA-CA)$, they will keep the
content from $(IA-CA)$. If when comparing the index with the current
commit it contains only new files and modified files (not deleted
files) then it is all. $INDEX = CB ++ (IA-CA)$. When the index
contains removed files, or in other words, files that are in $CA$ but are not in
$IA$ ($CA-IA$ is not empty), then case the file that is marked as removed exists
in CB it will be marked as removed, otherwise the removed information is
ignored. So, the result of performing a checkout is: $INDEX = (CB ++
(IA-CA)) - (CA-IA)$. All files from $CB$ are overwritten by the files
that are in the index but not in CA and from the result of this, the
files that are marked as removed will not be in index.

\subsubsection{Examples}
As referred above checkout behaves in a non-intuitive way. We recommend that
before performing a checkout, the user ensures that all the files from
the index are committed, otherwise there is a risk to pass content to other branches or even
to loose information. \\

Thus, the example shown in figure \ref{fig:checkout}, contains an
example in which all content in the index and working directory are committed (the safe case).
When performing a checkout we can see in this figure 
that the repository and index are updated to reflect the branch checked out, more
concretely the commit pointed by the branch. Something the reader can
also observe is that, after the checkout is performed, the HEAD
identifies the branch that was checked out.\\


\begin{figure}[!t]
   \centering
   \includegraphics[width=0.6\textwidth]{images/checkout.png}
   \caption{A simple checkout}\label{fig:checkout}
\end{figure}

\subsection{Merge}

The merge operation is one the of most important operation in \emph{git}. It
consists on trying to combine information from two commits. The result of that
is a new commit with information from both commits. However there are some cases
where this is not true. One of this cases is when the current commit
is a precedent from the commit being merged. In this case, it is not
created a new commit. The current branch (identified by HEAD) is
updated to point to the commit being merged.\\

Sometimes this operation of joining commits is not possible to do
automatically because there are conflicts, or in other words, when
the same file exists in both commits it is
not possible to say which one is more recent. In this case the merge
operation has to be finished by the user. The user has to resolve the
conflicts and then perform a commit.\\

\emph{Git} tries to be as most autonomous as possible when merging.
Thus, it has three types of merge:

\begin{itemize}
   \item \textbf{Fast-forward merge} (figure \ref{fig:fast_forward}):
   It happens when the commit being merged, can achieve the current commit by the 
   parent relation. If that happens, then the commit being merged is more recent
   than the current commit. The result will be the HEAD pointing to
   the commit being merged. It is not created a new commit; 
   
   \item \textbf{3-way merge} (figure \ref{fig:3mergecase}): 
   It happens when a fast-forward is not possible, but, both commits have a common
   ancestor. \emph{Git} will use the common ancestor to minimize potential
   merge conflicts.
   
   \item \textbf{2-way merge} (figure \ref{fig:2mergecase}): 
   It happens when both types above are not possible. In this case there are no
   common ancestor, thus it will be more likely that conflicts will
   appear.
\end{itemize}

\begin{figure}[tp]
   \centering
   \includegraphics[width=0.5\textwidth]{images/fast_forward.png}
   \caption{A fast-forward case}\label{fig:fast_forward}
\end{figure}

\begin{figure}[tp]
   \centering
   \includegraphics[width=0.6\textwidth]{images/3merge.png}
   \caption{A 3-way merge case}\label{fig:3mergecase}
\end{figure}

\begin{figure}[tp]
   \centering
   \includegraphics[width=0.6\textwidth]{images/2merge.png}
   \caption{A 2-way merge case}\label{fig:2mergecase}
\end{figure}

\subsubsection{Pre requisites}
The merge operation, as all other operations, has some pre conditions that
must be satisfied in order to perform it. As in the checkout operation,
\emph{git} lets the user to perform a merge without all information in the index 
being committed: 

\begin{itemize}
   \item A merge cannot be performed if the current commit is more
   recent than the commit being merged. As the current commits is more
   recent, no changes would exist. 
   
   \item There cannot exist an unfinished merge on the repository. If
   the user performed a merge previously and that merge was not
   automatic, then a commit must be performed before performing a new
   merge.
   
   \item When it is detected that the fast-forward merge will take
   place, the index cannot have uncommitted files that would would
   conflict with files from the resulting merge.
   
   \item When it is detected that a 2-way or 3-way merge will take
   place, then it is not possible to exist uncommitted files in index.
\end{itemize}

\subsubsection{Result}
As already said when performing a merge, one of three tactics can be
applied. In the case of a fast-forward merge, the operation will be
always automatic. It means that no conflicts will ever be found. In
the case of a 2-way merge or 3-way merge, the merge operation sometimes cannot
be resolved automatically because conflicts are found. Next, we give a
small description of what changes after a merge operation is called.

\begin{itemize}
\item If it is the case of a fast-forward merge. The index and the working directory
will contain the information of the newest commit. Also, they will contain all 
information prior to the merge that was uncommitted, as long as there
are no conflicts. Of course, the current branch and HEAD will point to the
newest commit.

\item When it is a 2-way or 3-way automatic merge, the
index and the working directory will reflect the new
commit (the created commit). Here, the question of 
uncommitted changes does not exists because \emph{git} 
does not allow uncommitted files when applying this tactics. 
In the repository a new commit is created that has the combined
changes of the two (possibly three, if it is a 3-way merge) 
commits involved in the merge. That new commit will have as 
parents those two commits and it will be pointed by the
branch identified by the HEAD.

\item If when performing a merge, it cannot be performed
automatically, git will throw a message saying that the 
automatic merge failed and the user has to finish it. In 
other words it leaves to the user the work of resolving the conflicts
that \emph{git} could not resolve. At that moment, the index and working directory
will contain all the files that were merged. The working directory
will also contain the unmerged files, with marks for the user to solve
the conflicts. The index contains the 2 (or 3 in the case of a 3-way
merge) versions of the files. As soon as the user solves the conflicts
and adds it to the index, this file will be marked as merged. After
all files are merged, a commit can be performed. In this manual
process, git keeps somewhere in the repository that a manual merge is
taking place and it also keeps which commits are being merged. This
information is used when the commit operation is performed, mainly to
know which commits will be the parent from the created one. 

\end{itemize}

\subsubsection{Examples}

As usual the next figures will show some concrete examples of what happens
in \emph{git} when a merge is done. We do not involve here cases of 
uncommitted files, as we do not recommend that and 
it would turn much more tricky to understand and much less intuitive. \\

Figure \ref{fig:ffmerge1} is a simple case of a merge operation with
the branch ''Foo''. The figure
shows that the Master branch (which is identified by HEAD) points to
a commit that is achievable by the commit pointed by 'Foo' and the
parent relation. So this is
a case eligible for a fast-forward commit.  We can see that the result
is that the working directory and index will be updated accordingly to
the new created commit now pointed by 'Foo' and Master.\\

\begin{figure}[htp]
   \centering
   \includegraphics[width=0.6\textwidth]{images/fast_forward_merge1.png}
   \caption{A fast-forward merge case}\label{fig:ffmerge1}
\end{figure}

Figure \ref{fig:ffmergepre} shows a concrete example of a state in
which it is not possible to perform a merge. This happens because 
the commit pointed by Master (which is identified by HEAD) takes
precedence over all files. \\

\begin{figure}[htp]
   \centering
   \includegraphics[width=0.6\textwidth]{images/fast_forward_merge_pre.png}
   \caption{A fast-forward case that can't be done in
   \emph{git}}\label{fig:ffmergepre}
\end{figure}

The two last figures \ref{fig:merge2way} and \ref{fig:merge3way} demonstrates 
the difference between a 2-way and a 3-way merge commit. When using
2-way merge there
are conflicts that cannot be solved automatically, in the 3-way merge
the conflicts are resolved automatically by looking at
the common ancestor. Because the 3-way merge was automatic, a new commit was
created that represents the merged information. Also, the working directory and
index were updated accordingly. In the 2-way merge the system is left
in a state for the user to solve the conflicts manually by updating
the unmerged files and adding it to the index.
Finally, when 'git commit' is performed a new merge commit is created and 
the merge process terminates.\\

\begin{figure}[htp]
   \centering
   \includegraphics[width=0.7\textwidth]{images/merge2way}
   \caption{A typical 2-way merge}\label{fig:merge2way}
\end{figure}

\begin{figure}[htp]
   \centering
   \includegraphics[width=0.7\textwidth]{images/merge3way}
   \caption{A typical 3-way merge}\label{fig:merge3way}
\end{figure}

\newpage
\subsubsection{Git Pull and Git Push}

When working with remote repositories, these two operations will be vastly
used. They are in fact, a sequence of operations and they are used to get content
from a remote repository or to push local content to a remote repository.\\

If a pull operation is performed, the branch from the remote repository will be fetched
and it will be merged with the local branch. So, it is expected from
this operation to behave exactly as a merge operation and consequently
everything that was referred in the merge operation applies also here. \\

If a push operation is performed, the local branch will be uploaded to
a remote repository. A pre-condition of this operation is that the
local branch contains the remote branch. This happens to ensures that
the local user resolves the conflicts.
