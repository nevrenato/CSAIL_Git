\chapter{Properties}
After the \emph{git} internals and some operations were modeled,
we checked for some properties. These properties are important to
prove the correct behavior of the model. Such properties were checked using
Alloy Analyzer, which tries to find counter examples for the
properties. This analysis is bounded, which means that it is not a
complete analysis, but instead it analyses the model up to
\emph{''n''} objects.\\

In the next sections we present some properties checked. We start with
the invariant preservation, then we show which operations are
idempotent and then we check for operation's specific properties.

\section{Invariant Preservation}
An invariant is a property that is always true in the model. In this
section, we show that the operations we modeled preserve an invariant
when they are performed. Our invariant says the following:
\begin{enumerate}
   \item There is a commit if and only if there is some created branch
   and HEAD is known;
   %some Commit & objects.s <=> some marks.s && one head.s
   \item The HEAD identifies an existing branch;
   \item Every existing branch points to an existing commit;
   %head.s in branches.s & (marks.s).Commit
   \item Each branch identifies only and only one commit;
   %marks.s in branches.s -> one objects.s
   \item There is not any object that does not descend from an existing
   commit;
   %(objects.s - Commit) in (objects.s).points.*(contents.Name)
   \item Every Tree has content (Blob and Trees) and these objects
   exists in the repository;
   \label{item:treeContainsSome}
   %all t : objects.s & Tree | t.contents.Name in objects.s
   %all s:State, t : objects.s & Tree | some t.contains
   \item Every commit object points to an existing tree and if it has
   a parent it exists in the repository;
   %all c:objects.s & Commit | c.points in objects.s and c.parent in objects.s
   \item There is not two consecutive commits pointing to the same tree.
   %all c,c': Commit & objects.s | c' in c.parent => c.points != c'.points  
\end{enumerate}

The most of the properties are trivial. The first property exists,
because before the first commit is created, it is not possible to
create branches and after that, it is never possible to remove the branch
identified by HEAD. The property number 
\ref{item:treeContainsSome} exists, because
\emph{git} does not allow empty Trees and every object contained in a
tree exists in the repository. The last property guarantees
that each existing commit has a different snapshot when comparing to
its parent.\\

To check if an operation preserves an invariant, the invariant must be
valid before performing the operation and after it is performed, the
invariant must still be valid. We have checked for each operation and
we conclude that our operations preserve the invariant.


\section{Idempotent Operations}
An operation is said idempotent if after being performed, repeating it
does not changes the state of the system. It does not make sense to
check if this property holds for some operations. For example after
removing a file or a branch, it cannot be removed again. The
idempotent operations in our model are: commit, add, and checkout.
Next we try to justify such property.\\

The commit operation is idempotent because, if we perform a commit which 
is equal to the previous, git will not create a new commit (not
allowed) and anything is changed in the repository.\\

The add operation is also idempotent because while the file in the
working directory keeps its content the add operation, after performed
the first time, does not brings any differences.\\

The last operation we proved to be idempotent and maybe the less
intuitive is the checkout. As we have seen in
\ref{sec:checkout}, the checkout operation updates the index and
the working directory to reflect the commit pointed by the branch being
checked out. Even if there are some ''weird'' pre-conditions this
operation is still idempotent.

\section{Add, Remove and Commit Operations}
For this operations we have checked the following properties, which
were valid. To notice that when we write ''new file'', we mean a file
that does not exists in index.
\begin{enumerate}
   \item Adding a new file to the index and removing it, brings us to the
   initial state;
   \item If a commit is performed after an add operation, the added
   file with the respective content will be in the commit pointed 
   by the branch identified by HEAD;
   \item If a commit is performed after an remove operation, the
   removed file will not be present in the commit pointed by the
   branch identified by HEAD;
   \item If a commit is performed, a new file is added to index and a
   new commit is performed, then the first commit and the second
   commit will point to different Trees;
   \item Assuming that the invariant is verified and the following operations are
   performed: commit, add a new file, commit, remove the file added
   previously, and commit. The first commit created and the last one
   will point to the same Tree object;
\end{enumerate}


\section{Branch and Checkout Operation}
For the branch operations (add and remove) we have checked two
properties. We have checked if after creating a branch and then
removing it we reach a state with the same branches as in the begin.
We have also checked if when a branch is created it points to the same
commit as the branch identified by HEAD. In both cases no counter
examples were found.\\

Relatively to the checkout operation it was shown the following:
\begin{enumerate}
   \item If the invariant is true, a checkout from a branch is
   performed and after this a checkout from the initial branch is
   performed then we reach a state where the HEAD identifies the same
   branch as in begin;
   \item If a commit is performed, a checkout from a random branch is
   performed and after that a new checkout from the initial branch is also
   performed, then the index content after the commit will be the same
   as in the end.
\end{enumerate}

Besides this two properties, we have also checked: if when the invariant is
true and a checkout from a branch ''b'' is performed, then all content
from the commit pointed by ''b'' will be in index. Such property is
not true, due, the strange pre-condition we wrote in
\ref{sec:checkout:pre}.
