%\subsection{Introducing branches}
%Definition of branch: "A branch in Git is {\bf simply a 
%lightweight movable pointer to one of these commits}." \cite{progit} 
%(pag 39)\\
%
%\begin{lstlisting}
%sig Branch{
%   on: Commit
%}
%\end{lstlisting}
%
%A branch in git, is a pointer for a commit. We are always 
%working on a certain branch, and all the commits made on 
%that branch are independent from the others branch until we 
%make a merge. A new branch can be created using 
%\emph{git branch branch\_name}. The special pointer called Header 
%points to the branch we are working on. \cite{progit} (pag 40)
%% Especifica qual é o quote 
%
%\begin{lstlisting}
%one sig Head{
%   current: Branch
%}
%\end{lstlisting}
%
%We can navigate by different branches using 
%\emph{git checkout branch\_name}.\\


%Some rules
%Trying to understand branches using git:
%\begin{itemize}
%   \item if we modify a tracked file, we cannot change branch if we do not add that file to the index and commit it;
%%   \item if we add a tracked and modified file to the index and we change branch, that file will be presented on the working directory;
%   % Nao percebo esta de baixo
%   \item if we create a new file on a branch and then we change branch (without adding the file to the index) that file will be present;
%   \item if we create a new file, add it to the index, change branch, commit, go back to the other branch, that file will not be visible;
%   \item master branch can be deleted;
%\end{itemize}
%
%% Entao como é q constroi a árvore ?
%We can conclude that the index is not connected to a branch. It is something is common to all branches.

