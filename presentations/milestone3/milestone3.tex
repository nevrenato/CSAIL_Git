\documentclass{beamer}
\usetheme{Singapore}
\usecolortheme{default}

\usepackage[utf8]{inputenc}
\usepackage[T1]{fontenc}
\usepackage{verbatim}
\usepackage{graphics}
\usepackage{listings}
\usepackage{lmodern}

\title{Understanding Git with Alloy}
\subtitle{Milestone 3}
\author{Cláudio Lourenço \and Renato Neves}
\institute{University of Minho\\
Formal Methods in Software Engineering}


\logo{ \includegraphics[width=0.15\textwidth]{images/csail_logo.png}
       \includegraphics[width=0.15\textwidth]{images/uminho_eng_logo.png}}

\begin{document}

\frame {
   \titlepage
}

\frame{
   \frametitle{Table of contents}
   \tableofcontents 
}

\section{Git as VCS}

\begin{frame}
	\frametitle{Git as VCS}
	\begin{block}{Git is one of many Version Control Systems}
		\begin{itemize}
		\item Fast
		\item Efficient
		\item Oriented to snapshots, not differences
		\item Widely used
	\end{itemize}
	\end{block}
\end{frame}

\section{Project motivation and objectives}
\begin{frame}
	\frametitle{Motivation for this project}
	\begin{block}{Gap in the understanding of Git}
		\begin{itemize}
		\item Lack of precise descriptions
		\item Contradictions in some manuals
		\item Developers could benefit from a manual
		that is precise and rigorous
		\end{itemize}
	\end{block}
\end{frame}

\begin{frame}
	\begin{itemize}
	\item "if there are any uncommitted changes when you run git checkout,
	 Git will behave very strangely." \footnote{Understanding Git}
	\item "When you create a branch, it will contain everything
         committed on the branch you created it from at that given
         point. So if you commit more things on the master branch like
         you have done (after creating b), then switch to branch b,
         they won't appear. This is the correct behavior. Does that
         answer your question?" \footnote{An average user of Git} 
	\end{itemize}
\end{frame}

\begin{frame}
	\frametitle{Objectives of this project}
	\begin{block}{Shine some light in the dark world of Git}
		\begin{itemize}
			\item Build a precise model of how Git works
			\item Analyze the model
			\item Build a user manual based on specification a
			analysis 
		\end{itemize}
	\end{block}
\end{frame}


\section{Git internals}

\begin{frame}
   \frametitle{The Git Structure}
   \begin{figure}
      \centering
      \includegraphics[width=0.5\textwidth]{images/git_workflow.png}
   \end{figure}
\end{frame}


\begin{frame}
\frametitle{Repository}
   \begin{figure}
      \centering
      \includegraphics[width=0.5\textwidth]{images/legenda2.png}
   \end{figure}
\end{frame}

\begin{frame}[fragile]
   \frametitle{Blob and Tree}
   \begin{block}{Blob}
      \begin{itemize}
         \item Represents the content of a file;
         \item The name is calculated from its content;
      \end{itemize}
      \tiny
      \color{blue}
      \begin{lstlisting}
      sig Blob extends Object {}
      \end{lstlisting}
   \end{block}
   \begin{block}{Tree}
      \begin{itemize}
         \item Relation from names to Blobs or/and Trees;
         \item Used to represent the file system structure;
      \end{itemize}
      \tiny
      \color{blue}
      \begin{lstlisting}
      sig Tree extends Object {
         
         contains: Name -> lone(Tree+Blob)
      
      }
      \end{lstlisting}
   \end{block}
\end{frame}




\begin{frame}[fragile]
   \frametitle{Commit}
   \begin{itemize}
      \item It is like a snapshot of the project on a certain moment
      in time;
      \item Author, Committer, Comment - Not important for us;
      \item Parent - The Commit which originated the current;
      \item Tree - Pointer to a Tree Object;
   \end{itemize}
   \tiny
   \color{blue}
   \begin{lstlisting}
                        sig Commit extends Object {
                           points : Tree,
                           parent : set Commit,
                           abs: Path -> Object,
                           merge : set State
                        }
                           
                        sig RootCommit extends Commit {}
\end{lstlisting}
\end{frame}

\begin{frame}[fragile]
\frametitle{Branch and HEAD}
   \begin{block}{Branch}
      \begin{itemize}
         \item It is just a pointer to a commit;
      \end{itemize}
   \end{block}
   \begin{block}{HEAD}
      \begin{itemize} 
         \item Special reference that identifies the current Branch;
      \end{itemize}
   \end{block}
   \tiny
   \color{blue}
   \begin{lstlisting}
                        sig Branch{
                           marks: Commit lone -> State,
                           branches: set State,
                           head: set State
                        }

                        lone sig Master extends Branch{}
   \end{lstlisting}
\end{frame}

\begin{frame}
	\frametitle{Repository}
	\begin{figure}
		\centering
		\includegraphics[width=0.5\textwidth]{images/object_assoc.png}
	\end{figure}
\end{frame}

\begin{frame}[fragile]
   \frametitle{Working Directory}
   \begin{itemize}
      \item Subset of a file system with the content of a project;
      \item These files can be the current files or files retrieved
      from the repository.
   \end{itemize}
   \tiny
   \color{blue}
   \begin{lstlisting}
                        sig Path {
                           pathparent: lone Path,
                           name: Name,
                           unmerge: set State
                        }

                        one sig Root extends Path{}
   \end{lstlisting}
\end{frame}

\begin{frame}[fragile]
   \frametitle{Index}
   \begin{itemize}
      \item Something in between the working directory and repository;
      \item It keeps a relation from file to content;
      \item The files in index will be in the next commit;
   \end{itemize}
   \vspace{10mm}
   \tiny
   \color{blue}
   \begin{lstlisting}
                        sig File{
                           path: Path,
                           blob: Blob,
                           index: set State
                        }

   \end{lstlisting}

\end{frame}


\section{Specification of operations}

\begin{frame}[fragile]
   \frametitle{Add}
   \begin{figure}
      \centering
      \includegraphics[width=0.3\textwidth]{images/add1.png}
   \end{figure}
\end{frame}

\begin{frame}[fragile]
   \frametitle{Remove}
   \begin{figure}
      \centering
      \includegraphics[width=0.45\textwidth]{images/remove1.png}
   \end{figure}
\end{frame}

\begin{frame}[fragile]
   \frametitle{Commit}
   \begin{figure}
      \centering
      \includegraphics[width=0.45\textwidth]{images/commit1.png}
   \end{figure}
\end{frame}

\begin{frame}[fragile]
   \frametitle{Commit}
   \begin{figure}
      \centering
      \includegraphics[width=0.45\textwidth]{images/commit2.png}
   \end{figure}
\end{frame}

\begin{frame}[fragile]
   \frametitle{Checkout}
   \begin{figure}
      \centering
      \includegraphics[width=0.45\textwidth]{images/checkout.png}
   \end{figure}
\end{frame}

\begin{frame}[fragile]
   \frametitle{ A fast-forward Merge}
   \begin{figure}
      \centering
      \includegraphics[width=0.45\textwidth]{images/fastforwardmerge.png}
   \end{figure}
\end{frame}

\begin{frame}[fragile]
   \frametitle{A 2-way Merge}
   \begin{figure}
      \centering
      \includegraphics[width=0.45\textwidth]{images/merge2way.png}
   \end{figure}
\end{frame}

\begin{frame}[fragile]
   \frametitle{Modeled Operations}
   \begin{itemize}
      \item Add and Remove
      \item Commit
      \item Branch and Branch Remove
      \item Checkout
      \item Merge (2-way and fast-forward)
   \end{itemize}
\end{frame}

\section{Documentation}

\begin{frame}
	\frametitle{Manual}
	\begin{block}{Built a manual that describes}
	\begin{itemize}
		\item Git internals
		\item Git operations
	\end{itemize}
	\end{block}
\end{frame}

\begin{frame}
	\frametitle{Website}
	\begin{itemize}
	\item Website created based on the manual 
	\item http://nevrenato.github.com/CSAIL\_Git
	\end{itemize}

\end{frame}

\section{Conclusion}

\begin{frame}
	\frametitle{Future Work}
	\begin{itemize}
	\item Model more operations (rebase, fetch, 3-way merge...) 
	\item Specify more properties that the model does (not) guarantee
	\item Build interactive diagrams of concrete examples of operations 
	\end{itemize}
\end{frame}

\begin{frame}
	\frametitle{Conclusions}
\end{frame}


\end{document}
